\unit{ $Metric Continuity$ }
{
	\introduction{ 
		$
		The field of real numbers is the transcendent extension \\
		of the rational numbers that represent all the possible \\
		values that can be obtained once a reference is set in a \\
		line. This set and its properties have been used since the \\
		very beggining of the mathematics but its formalization didn't \\
		come off till the 20th century. As we will discuss in this unit, \\
		the field of real numbers is the defined by a field that \\
		accomplishes some special properties. The point is that exists \\
		one unique field ( over isomorphisms ) that has this properties \\
		and thats why its calles The field of real numbers. Lets start \\
		From the very beggining:
		$
	}
	\definition{ $Limit$ }
	{
	\letbe
	{
		(A_1,d_1), (A_2,d_2) $ metric spaces $.
		\function{ f }{ A_1 }{ A_2 }.
		B \subset A_1.
		x \in B'.
		l \in A_1

	}
	\then{ l }{ $the limit of f in $ x }
	{
		\all{ \eps \in \R^+ }
		{
			\ex{ \delta \in \R^+ }{ \all{ y \in B }[ d_1(x,y) < \delta ]
			{
				d_2( l, f(y) ) < \eps
			} }
		}
	}
	\denote
	{
		l : \limit{ f(t) }{ t }[ x ]
	}
	}


	\definition{ $Continuity$ }
	{
	\letbe
	{
		(A_1,d_1), (A_2, d_2) $ metric spaces $.
		B \subset A_1.
		\function{ f }{ B }{ A_2 }.
		x \in B'
	}
	\then{ f }{ $continuous in $ x }
	{
		f(x) = \limit{ f(t) }{ t }[ x ]
	}
	\then{ f }{ $continuous in $ B }
	{
		\all{ x \in B }
		{
			f $ continuous in $ x
		}
	}
	\denote
	{
		f $ continuous in $ A \as f \in \Cc(A_1)
	}}



	\definition{ $Uniform Continuity$ }
	{
	\letbe
	{
		(A_1,d_1), (A_2,d_2) $ metric spaces $.
		B \subset A.
		\function{ f }{ B }{ A_2 }
	}
	\then{ f }{ $uniformly continuous in $ B }
	{
		\all{ \eps \in \R^+ }
		{
			\ex{ \delta \in \R^+ }{ \all{ x,y \in B }[ d_1(x,y) < \delta ]
			{
				d_2(f(x),f(y)) < \eps
			} }
		}
	}
	}

}