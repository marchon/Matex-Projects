
\documentclass[12pt]{book}

\usepackage{matex}

\begin{document}

\laboratory{ $Logarithm determinations$ }
{
	\baselineskip=35pt
	\proposition{ $ Martin Azpillaga $ }
	{
		\letbe
		{
			\definedFunction{ log }{ \mathbb{\C} \setminus (-\infty,0] }{ \mathbb{\C} }{ 1 }{ 4\pi i } $ logarithm determination $.
			\definedFunction{  f  }{ \mathbb{\C} \setminus [1,\infty) }{ \mathbb{\C} }{ z }{ -log(2-2z) }.
			\all{ z \in \D(\frac{ 1 }{ 2 },\frac{ 1 }{ 2 }) }
			{
				S(z) = \summation{ \frac{ (2z-1)^n }{ n } }{ n }[ 1 ]
			}
		}
		\holds
		{
			f \in \Hc( \mathbb{\C} \setminus [1,\infty) ).
			f = S + 4\pi i $ over $ \D(\frac{ 1 }{ 2 },\frac{ 1 }{ 2 })
		}
		\demonstration
		{
			\definedFunction{ g }{ \mathbb{\C} \setminus [1,\infty) }{ \mathbb{\C} \setminus (-\infty,0] }{ z }{ 2-2z }.

			g $ well defined $:.

			\all{ z \in \mathbb{\C} \setminus [1,\infty) }[Im(g(z)) = 0 ]
			{
				Im(2-2z) = 0 \imp -2Im(z) = 0 \imp Im(z) = 0 \imp Re(z) < 1.
				Re(g(z)) = Re(2-2z) = 2 -2Re(z) > 0
			}.
			g \in \polynomial( \mathbb{\C} \setminus [1,\infty) ) \imp g \in \Hc( \mathbb{\C} \setminus [1,\infty) ).

			log \in \Hc( \mathbb{\C} \setminus (-\infty,0] ).

			f = -log \circ g \in \Hc( \mathbb{\C} \setminus [1,\infty) ).

			$In particular$:.

			log(1) = 4\pi i \imp ln(1) + i arg(1) = 4\pi i \imp arg(1) = 4\pi i.

			f(0) = -log(2) = -ln(\abs{ 2 }) - i\arg(2) = -ln(2) - i\arg(1) = -ln(2) - 4\pi i.
			f(-i) = -log(2+2i) = -ln(\abs{ 2+2i }) -i\arg( 2 + 2i) = -ln(\sqrt{ 8 }) -i(\arg(1) + \frac{ \pi }{ 2 }) = -ln(\sqrt{ 8 }) - i(\frac{ 17\pi }{ 4 })..

			S $ well defined:$.
			\summation{ \frac{ (2z-1)^n }{ n } }{ n }[ 1 ] = \summation{ \frac{ 2^n(z- \frac{ 1 }{ 2 })^n }{ n } }{ n }[ 1 ].

			\limit{ \frac{ \frac{ 2^{n+1} }{ n+1 } }{ \frac{ 2^{n} }{ n } } }{ n } = \limit{ \frac{ 2(n+1) }{ n } }{ n } = 2.

			$Quotient test:$.

			\limit{ (\frac{ 2^n }{ n })^{-\frac{ 1 }{ n }} }{ n } = 2^{-1} = \frac{ 1 }{ 2 }.

			$Cauchy-Hadamard theorem:$.

			S(z) $ convergent over $ \D(\frac{ 1 }{ 2 },\frac{ 1 }{ 2 }).

			$Sum of $ S:.

			\be{ x }{ z- \frac{ 1 }{ 2 } }.

			\summation{ \frac{ (2z-1)^n }{ n } }{ n }[ 1 ] = \summation{ \frac{ 2^nx^n }{ n } }{ n }[ 1 ] = \summation{ \frac{ (2x)^n }{ n } }{ n }[ 1 ].

			$UCD theorem$:.

			S'(z) = \summation{ 2^nx^{n-1} }{ n }[ 1 ] = 2\summation{ (2x)^{n-1} }{ n }[ 1 ] = \frac{ 2 }{ 1-2x }.

			S \in \int \frac{ 2 }{ 1-2x } dx = \family*{ -log(1-2x)+c }{ c \in \mathbb{\C} }.

			S(0) = 0 \imp S = -log(1-2x) = -log(2-2z).

			\D(\frac{ 1 }{ 2 },\frac{ 1 }{ 2 }) \cap \mathbb{\C} \setminus (-\infty,0] = \emptyset \imp S(z) = -Log(2-2z).
			
			$log and Log relationship:$.

			log, Log $ well defined over $ \mathbb{\C} \setminus (-\infty,0].

			\all{ z \in \mathbb{\C} \setminus (-\infty,0]}
			{
				log(z) - Log(z) = ln(z) + i arg(z) - ln(z) -iArg(z) = i(arg(z) -Arg(z))=.
				= i((Arg(z)+arg(1))- Arg(z)) = 4\pi i.

				log(z) = Log(z) + 4 \pi i
			}.

			\all{ z \in \D(\frac{ 1 }{ 2 },\frac{ 1 }{ 2 }) }
			{
				S(z) - f(z) = -Log(2-2z) + log(2-2z) = i(arg(2-2z) - Arg(z)).

				z \in \C \setminus (-\infty,0] \imp S(z) - f(z) = 4 \pi i.

				S(z) = f(z) + 4\pi i
			}

		}
	}
}



\end{document}