



\documentclass[../Main/main]{subfiles}


\begin{document}


\unit{ $ New $ }
{

	\proposition{ $ logarithm is holomorphic $ }
	{
		\letbe
		{
			\function{ log }{ \C \setminus e^{i\alpha}(-\infty,0] }{ B_\alpha }
		}
		\holds
		{
			log \in \Hc.
			log' = \frac{ 1 }{ z }
		}
		\demonstration
		{
			log $ continuous $.
			\exp( \log( z ) ) = z.
			\omega = \log( z ).
			\omega_0 = \log( z_0 ).
			\exp( \omega ) = z.
			\exp( \omega_0 ) = z_0.
			log $ continuous $ \imp w \convergesto w_0 $ if $ z = z_0.
			\frac{ log(z) - log(z_0) }{ z - z_0 } = \frac{ w- w_0 }{ e^w - e^{w_0} }.
			 = \limit{ \frac{ 1 }{ \frac{ e^w - e^{w_0} }{ w - w_0 } } }{ w }[ w_0 ] = \frac{ 1 }{ e^{w_0} } = \frac{ 1 }{ z_0 }
		}
	}
	
	
	\proposition{ $ Complex Powers $ }
	{
		\letbe
		{
			z^w = e^{wlog z} = e^{w(log|z| + iarg(z)}
		}
		\holds
		{
			z^w = \exp( n \log( z ) )
		}
		\demonstration
		{
			z = \exp( \log( z ) ).
			z^w = \exp( \log( z ) )^n = \product{ \exp( \log( z ) ) }{ i }[ 1 ][ n ] = \exp( n \log( z ) )
		}
	}
	
	
	\example{ $ i powers $ }
	{
		\letbe
		{
			i \in \C
		}
		\is{ i^i }{ $ an infinite value set $ }
		{
			\all{ k \in \Z }
			{
				i^i = \exp( i \log( i ) )=\exp( i^2(\frac{ \pi }{ 2 } + 2k\pi) ) = \exp( -(\frac{ \pi }{ 2 } + 2k\pi) )
			}
			
		}
	}
	
	
	\proposition{ $ Different values of powers $ }
	{
		\letbe
		{
			z,w \in \C
		}
		\holds
		{
			w \in \Z \imp z^w $ unique $.
			w = \frac{ p }{ q } \in \Q \imp z^w $ has q values $.
			w \in \R \setminus \Q \imp z^w $ has infinite values $ 
		}
	}
	
	
	\example{ $ Different logarithm definitions $ }
	{
		\letbe
		{
			\Omega = \C \setminus \set{ (x,y) \in \C \with x < 0, y = 0 }.
			\Omega_2 = \C \setminus \set{ (x,y) \in \C \with x > 0, y = x }.
			\Omega_3 = \C \setminus \set{ (x,y) \in \C \with x \in (-1,0), y = 0 } \cup \set{ (x,y) \in \C \with x=-1,y\in(0,\pi) } \cup \set{ (x,y= \in \C \with x > -1, y = \frac{ \pi }{ 2 } }
		}
		\study
		{
			arg(i),arg(2i) $ in function of $ arg(1)
		}	
		\start
		{
			arg(1) = 0 \imp arg(i) = \frac{ \pi }{ 2 }.
			arg(1) = -2\pi \imp arg(i) = \frac{ -3\pi }{ 2 }.
			arg(i) = \frac{ -3\pi }{ 2 } \imp arg(1) = -2\pi..

			
			arg(1) = 0 \imp arg(i) = \frac{ -3\pi }{ 2 }.
			arg(1) = -2\pi \imp arg(\pi) = \frac{ -7\pi }{ 2 }..

			arg(1) = 0 \imp arg(i) = \frac{ \pi }{ 2 }.
			arg(1) = 0 \imp arg(2i) =\frac{ -3\pi }{ 2 }.
			arg(i) = \frac{ -3\pi }{ 2 } \imp arg(2i) = \frac{ -7\pi }{ 2 }
		}
	}
	
	
	\definition{ $ Simple fraction form $ }
	{
		\letbe
		{
			f \in \rational( \C )
		}
		\name{ $ simple fraction form of $ f }
		{
			\summation{ \frac{ a_i }{ (x-a_i) } }{ i }[ 1 ][ r ] = f
		}
	}
	
	
	
	
	
	
	

}


\end{document}