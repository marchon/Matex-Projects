\unit{ $ Random Variables $ }
{
	
	\introduction{ $ The study of probabilities was started by Blaise Pascal. $ }

	\definition{ $ Probability Space $ }
	{
	\letbe
	{
		\Omega $ set $.
		\Ac \sigma$-algebra$.
		\function{ \Pc }{ \Ac }{ [0,1] }
	}
	\then{ (\Omega, \Ac, \Pc) }{ $a probability space $ }
	{
		\Pc(\Omega) = 1.
		\all{ \family{ A_i }{ i }[ 0 ] \subset \Ac }[ \family{ A_i }{ i }[ 0 ] $ mutually disjoint $ ]
		{
			\Pc( \union{ A_i }{ i }[ 0 ] ) = \sumatory{ \Pc(A_i) }{ i }[ 0 ]
		}
	}}

	\definition{ $ Conditioned Probability $ }
	{
	\letbe
	{
		(\Omega,\Ac,\Pc) $ probability space $
		B \in \Ac
	}
	\call{ $ probability conditioned by $ B $ over $ (\Omega,\Ac,\Pc) }
	{
		\definedfunction{ \Pc|_B }{ \Ac }{ [0,1] }{ A }{ \frac{\Pc(A \cap B)}{\Pc(B)} }
	}}

	\definition{ $ Independency $ }
	{
	\letbe
	{
		(\Omega,\Ac,\Pc) $ probability space $
		\family{ A_i }{ i }[ 1 ][ n ] \subset \Ac 
	}
	\then{ \family{ A_i }{ i }[ 1 ][ n ] }{ $an independent family$ }
	{
		\all{ \family{ k_i }{ i }[ 1 ][ r ] \subset \indexes{0}{n} }
		{
			\Pc(\intersection{ A_{k_i} }{ i }[ 1 ][ r ]) = \productory{ A_{k_i} }{ i }[ 1 ][ n ]
		}
	}}

	\definition{ $ Random Variable $ }
	{
	\letbe
	{
		(\Omega,\Ac,\Pc) $ probability space $.
		\function{ X }{ \Omega }{ \R }
	}
	\then{ X }{ $a random variable $ }
	{
		\all{ A \in \B(\R) }
		{
			X^{-1}(A) \in \Ac
		}
	}}

	\definition{ $ Law $ }
	{
	\letbe
	{
		(\Omega,\Ac,\Pc) $ probability space $.
		\function{ X }{ \Omega }{ \R } $ random variable $
	}
	\call{ $ law of $ X }
	{
		\definedfunction{ \Pc_X }{ \B(\R) }{ [0,1] }{ A }{ \Pc(X^{-1}(A)) }
	}
	%\denote
	{
	%	\Pc_X((-\infty,x)) \as \Pc(X \leq x)
	}}	

	\definition{ $ Distribution Function $ }
	{
	\letbe
	{
		(\Omega,\Ac,\Pc) $ probability space $.
		\function{ X }{ \Omega }{ \R } $ random variable $
	}
	\call{ $ cumulative distribution function of $ X }
	{
		\definedfunction{ F_x }{ \R }{ [0,1] }{ x }{ \Pc(X \leq x) }
	}}

	\definition{ $ Discrete Random Variable $ }
	{
	\letbe
	{
		(\Omega,\Ac,\Pc) $ probability space $.
		\function{ X }{ \Omega }{ \R } $ random variable $
	}
	\then{ X }{ $a discrete random variable$ }
	{
		X(\Omega) \countable
	}}

	\definition{ $ Probability Mass function $ }
	{
	\letbe
	{
		(\Omega,\Ac,\Pc) $ probability space $.
		\function{ X }{ \Omega }{ \R } $ discrete random variable $
	}
	\call{ $probability mass function of $ X }
	{
		\definedfunction{ p }{ X(\Omega) }{ [0,1] }{ x }{ P(X=x) }
	}}
	\newpage
	\definition{ $ Density function $ }
	{
	\letbe
	{
		\function{ f }{ \R }{ \R }
	}
	\then{ f }{ $a density function$ }
	{
		f \geq 0.
		f \Rc(\R).
		\integral{ f(x) }{ dx }{ \R } = 1
	}}

	\definition{ $ Absolutely Continuous random variable $ }
	{
	\letbe
	{
		(\Omega,\Ac,\Pc) $ probability space $.
		\function{ X }{ \Omega }{ \R } $ random variable $.
		F_X $ distribution function of $ X
	}
	\then{ X }{ $an absolutely continuous random variable$ }
	{
		\ex{ f \in \functionspace{ \R }{ \R } }{ \all{ x \in \R }
		{
			F_x(x) = \integral{ f(x) }{ dx }{ -\infty }[ x ]
		} }
	}
	\denote
	{
		X $an absolutely continuous random variable$ \as X $ abs cont $
	}}

	\definition{ $ Expected Value $ }
	{
	\letbe
	{
		(\Omega,\Ac,\Pc) $ probability space $.
		\function{ X }{ \Omega }{ \R } $ discrete random variable $.
		p $ probability mass function of $ X.
		\function{ X' }{ \Omega }{ \R } $ abs cont $.
		f_X $ density function of $ X'
	}
	\then{ X }{ $expectable$ }
	{
		\sumatoryoverset{ |x|p(x) }{ x }{ X(\Omega) } \in \R
	}
	\call{ $expected value of $ X }
	{
		\sumatoryoverset{ xp(x) }{ x }{ X(\Omega) } 
	}
	\then{ X' }{ $expectable$ }
	{
		\integral{ |x|f(x) }{ dx }{ \R } \in \R
	}
	\call{ $expected value of $ X' }
	{
		\integral{ xf(x) }{ dx }{ \R }
	}
	\denote
	{
		$expected value of $ X \as E(X)
	}}

	\definition{ $ Varaible change $ }
	{
	\letbe
	{
		(\Omega,\Ac,\Pc) $ probability space $.
		\function{ X }{ \Omega }{ \R } $ abs cont $.
		I = \open{I} \subset \R \suchthat \Pc(X \in I ) = 1.
		J = \open{J} \subset \R.
		\function{ g }{ I }{ J }
	}
	\then{ g }{ $a variable change over $ X }
	{
		g $ biyective $.
		g,g^{-1} \in \Cc^1
	}}
}