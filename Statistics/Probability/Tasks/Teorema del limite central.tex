\documentclass[a5paper]{article}
\renewcommand{\baselinestretch}{1.5}
\usepackage{matex}
\usepackage{titlesec}
\usepackage[spanish]{babel}
\usepackage[utf8]{inputenc}
\usepackage{bbm}
\usepackage{courier}
\usepackage[pdftex]{graphicx} % PDFLaTeX
\DeclareGraphicsExtensions{.png,.pdf,.jpg}

\renewcommand{\C}{\mathcal{C}}
\newcommand{\ind}[1]{\mathbbm{1}_{#1}}
\titleformat{\section}{\bfseries\filcenter}{}{0pt}{\bfseries\filcenter}
\NewDocumentCommand{ \integral }{ m m m O{} }{ \displaystyle\int_{ #3 }^{ #4 } #1 #2 }




\begin{document}
\titlePage{ $ Simulación del teorema central del límite $ }{ $ Martin Azpillaga $ }
\renewcommand{\blockname}{ Simulación del teorema central del límite }

\abstract
{
	
	{\bf Teorema} del límite central

	Sea:
	\begin{itemize}
	\item $\family{ X_i }{ i }[ 1 ] $ variables aleatorias i.i.d con momento de segundo orden finito.
	\item $\mu = E[X]$
	\item $\sigma^2 = Var(X) > 0$
	\end{itemize}

	Entonces, se cumple:
	$$\frac{ \sumatory{ X_i }{ i }[ 1 ][ n ] - n\mu }{ \sigma\sqrt{n} } \overset{\mathcal{L}}{\longrightarrow} N(0,1)$$

	Este trabajo consiste en la simulación del teorema del límite central para tres leyes. Una ley con densidad simétrica:
	\begin{itemize}
	\item$ f_1(x) = c|\sin(x)| \ind{(0,3\pi)}$ \hspace{15pt} $ c = \frac{1}{6} $
	\end{itemize}
	Una ley con densidad no simétrica:
	\begin{itemize}
	\item $f_2(x) = c\sin(x) \ind{(0,1)}$ \hspace{15pt} $ c = \frac{1}{1-\cos(1)} $
	\end{itemize}
	Y la distribución de Cauchy ordinaria dada por la densidad:
	\begin{itemize}
	\item $f_3(x) = \frac{ 1 }{ \pi(1+x^2) }$
	\end{itemize}
}

\section{ Consideraciones previas }

Para calcular los momentos de primer y segundo orden, necesitaremos encontrar las siguientes primitivas:

$ \displaystyle\int x\sin(x) = \sin(x) - x\cos(x) + c$

$ \displaystyle\int x^2\sin(x) = -x^2\cos(x) + 2x\sin(x) + 2\cos(x) + c$

Llamamos:

$ g(x) = \sin(x) - x\cos(x) $

$ h(x) = -x^2\cos(x) + 2x\sin(x) + 2\cos(x) $


\section{ Ley con densidad simétrica }

{\bf Esperanza }

Veamos que $f_1$ tiene esperanza finita:

$\integral{ |x|f_1(x) }{ dx }{ \R } = c \integral{ x |\sin(x)| }{ dx }{ 0 }[ 3\pi ] = $

$c\left ( \integral{ x \sin(x) }{ dx }{ 0 }[ \pi ] - \integral{ x\sin(x) }{ dx }{ \pi }[ 2\pi ] + \integral{ x \sin(x) }{ dx }{ 2\pi }[ 3\pi ]\right ) = $

$ c\left ( [g(x)]_0^\pi - [g(x)]_\pi^{2\pi} + [g(x)]_{2\pi}^{3\pi} \right ) = 9\pi c$

Como $|x| = x$ en $(0,3\pi)$, $\mu = 9\pi c$.\\

{\bf Variancia }

Veamos que $f_1$ tiene momento de segundo orden finito:

$\integral{ |x^2|f_1(x) }{ dx }{ \R } = c \integral{ x^2 |\sin(x)| }{ dx }{ 0 }[ 3\pi ] = $

$c\left ( \integral{ x^2 \sin(x) }{ dx }{ 0 }[ \pi ] - \integral{ x^2\sin(x) }{ dx }{ \pi }[ 2\pi ] + \integral{ x^2 \sin(x) }{ dx }{ 2\pi }[ 3\pi ]\right ) = $

$ c\left ( [h(x)]_0^\pi - [h(x)]_\pi^{2\pi} + [h(x)]_{2\pi}^{3\pi} \right ) = (19\pi^2 -12)c $

Como $|x| = x$ en $(0,3\pi)$, $E[X^2] = (19\pi^2 -12)c$.

$\sigma^2 = (19\pi^2 -12)c - 81\pi^2c^2$\\

{\bf Función de distribución e inversa}

$F_1(x) = ((2[\frac{ x }{ \pi }]+1 + (-1)^{[\frac{ x }{ \pi }]+1}\cos(x))c)\ind{(0,3\pi)} + \ind{(3\pi,\infty)}$

$F_1^{-1}(y) = ([3y]\pi + \arccos(2[3y]+1 - \frac{ y }{ c }))\ind{(0,1)}$ en $(0,1)$


\section{ Ley con densidad no simétrica }

{\bf Esperanza }

Veamos que $f_2$ tiene esperanza finita:

$\integral{ |x|f_2(x) }{ dx }{ \R } = c \integral{ x \sin(x) }{ dx }{ 0 }[ 1 ] = $

$ c[g(x)]_0^1 = (\sin(1) - \cos(1)) c$

Como $|x| = x$ en $(0,1)$, $\mu = (\sin(1) - \cos(1)) c$.\\

{\bf Variancia }

Veamos que $f_2$ tiene momento de segundo orden finito:

$\integral{ |x^2|f_2(x) }{ dx }{ \R } = c \integral{ x^2 \sin(x) }{ dx }{ 0 }[ 1 ] = $

$ c [h(x)]_0^1 = (2\sin(1) - \cos(1))c $

Como $|x| = x$ en $(0,3\pi)$, $E[X^2] = (2\sin(1)+\cos(1)-2)c$.

$\sigma^2 = (2\sin(1)+\cos(1)-2)c - (\sin(1) - \cos(1))^2 c^2$.\\

{\bf Función de distribución e inversa}

$F_2(x) = (1-\cos(x))\ind{(0,1)}+\ind{(1,\infty)}$

$F_2^{-1}(y) = \arccos(1-y)\ind{(0,1)}$ en ${0,1}$

\section{ Ley de Cauchy }

{\bf Esperanza y Variancia}

$\integral{ |x|f_3(x) }{ dx }{ \R }$

$\integral{ \frac{ x }{ \pi(1+x^2) } }{ dx }{ \R^+ } = \lim_{x\to\infty} \integral{ \frac{ x }{ \pi(1+x^2) } }{ dx }{ 0 }[ \infty ] = $

$ \lim_{x\to\infty} \frac{ \ln(1+x^2) }{ 2\pi } \nin \R $

Por tanto, no tiene esperanza ni variancia.\\

{\bf Función de distribución e inversa}

$F_3(x) = \frac{ 1 }{ \pi }$arctg$(x) + \frac{ 1 }{ 2 }$

$F_X^{-1}(y) = \tan(\pi(y- \frac{ 1 }{ 2 }))$

\newpage



\section{ Anexo. Script R}

{\bf Ley con densidad simétrica}

\ttfamily{

N<-2000

n<-50

random<-runif(N*n)

dim(random)<-c(N,n)

inversa<-floor(3*random)*pi+acos(2*floor(3*random)+1 -6*random)

y<-apply(inversa,1,sum)

esperanza<-n*9*pi/6

desviacion<-(n*((19*pi\textasciicircum2 -12)/6 - 81*pi\textasciicircum2/36))\textasciicircum(1/2)

y<-(y-esperanza)/desviacion

hist(y,freq=FALSE,ylim=range(0,0.5))

x<-seq(-3,3,by=0.01)

normal<-dnorm(x,0,1)

lines(x,normal,col="blue")

}

{\bf Ley con densidad no simétrica}

\ttfamily{N<-2000
n<-50

random<-runif(N*n)

dim(random)<-c(N,n)

inversa<-acos(1-random)

y<-apply(inversa,1,sum)

esperanza<-n*(sin(1)-cos(1))/(1-cos(1))

desviacion<-(n*((2*sin(1)+cos(1)-2)/(1-cos(1)) - ((sin(1)-cos(1))\textasciicircum2)/((1-cos(1))\textasciicircum2)))\textasciicircum(1/2)

y<-(y-esperanza)/desviacion

hist(y,freq=FALSE,ylim=range(0,0.5))

x<-seq(-3,3,by=0.01)

normal<-dnorm(x,0,1)

lines(x,normal,col="blue")}

{\bf Distribución de Cauchy}

\ttfamily{N<-2000
n<-500

random<-rcauchy(N*n)

dim(random)<-c(N,n)

y<-apply(random,1,sum)

y<-y/n

hist(y,freq=FALSE,ylim=range(0,0.1))

x<-seq(-100,100,by=0.1)

normal<-dnorm(x,0,1)

lines(x,normal)}

\newpage

\section{ Resultados de la simulación }

\begin{figure}[h]
\centering
\includegraphics[scale=0.3]{simetrica.png}%ext=pdf,jpg,png
\caption{Ley con densidad simétrica. n=50}
\label{contexto:figura}
\end{figure}

\begin{figure}[h]
\centering
\includegraphics[scale=0.3]{cauchy.png}%ext=pdf,jpg,png
\caption{Distribución de Cauchy. n=500}
\label{contexto:figura}

\end{figure}

\newpage

\end{document}




%$\forall x \in (-\infty,0):$

% $F_1(x) = 0$

% $\forall x \in (0,\pi):$

% $F_1(x) = \integral{ c\sin(t) }{ dt }{ 0 }[ x ] = (1-\cos(x))c $

% $\forall x \in (\pi,2\pi):$

% $F_1(x) = (1-cos(\pi))c - \integral{ c\sin(t) }{ dt }{ \pi }[ x ] = (3 + \cos(x))c$

% $\forall x \in (2\pi,3\pi):$

% $F_1(x) = 4c + \integral{ c\sin(t) }{ dt }{ 2\pi }[ x ] = (5-\cos(x))c$

% $\forall x \in (3\pi,\infty):$

% $F_1(x) = 1$

% $F_1(x) = (2[\frac{ x }{ \pi }]+1 + (-1)^{[\frac{ x }{ \pi }]+1}\cos(x))c$

% $F_1^{-1}(y) = [3y]\pi + $arccos$(2[3y]+1 - \frac{ y }{ c })$

% $F_2(x) = 1-\cos(x)$

% $F_2^{-1}(y) = $arccos$(1-y)$
