
\documentclass[../Main/main]{subfiles}

\newcommand{\step}[2]{\end{math}\tabmath{$\textbullet #1$}{#2}\begin{math}}
\fancyhead[RE]{ Martin Azpillaga }
\fancyhead[LO]{ 1st laboratoy }
\begin{document}


\unit{ $ 1st laboratory $ }
{
	\proposition{ $ Orbit analysis $ }
	{
		\letbe
		{
			\definedFunction{ f }{ \R }{ \R }{ x }{ x^3 + \frac{ 1 }{ 4 }x }
		}
		\study
		{
			$Orbit behavior of the real dynamical system defined by $ f
		}
		\demonstration
		{
			\step{ $ Formalization $ }
			{
				$Consider $ (M,T,\phi) $ where:$.
				M = \R.
				T = \N.
				\definedFunction{ \phi }{ \R \times \N }{ \R }{ (x, n) }{ f^n(x) }.

				$Study the orbits of $ (\R,\N,\phi).
				
				$We will denote $ f^n(x) $ as $ x_n
			}.
			
			\step{ $ Fixed points $ }
			{
				\all{ x \in \R }
				{
					f(x) = x \ifandonlyif x^3 + \frac{ 1 }{ 4 }x - x = 0 \ifandonlyif x^3 - \frac{ 3 }{ 4 }x = 0.

				\ifandonlyif x = 0 \logicOr x^2 - \frac{ 3 }{ 4 } = 0.

				x $ fixed point $ \ifandonlyif x \in \set{ 0, \pm \frac{ \sqrt{ 3 } }{ 2 } }
				}.

			}.			


			\step{ $ Graphic analysis $ }
			{
				$ Parity:$.

				\all{ x \in \R }
				{
					f(-x) = (-x)^3 + \frac{ (-x) }{ 4 } = -( x^3 + \frac{ x }{ 4 }) = -f(x)
				}.

				f $ is odd $.

				$ Monotonicity: $.

				\all{ x \in \R }
				{
					f'(x) = 3x^2 + \frac{ 1 }{ 4 } > 0
				}.

				f $ is increasing over $ \R.

				$ Convexity:$.

				\all{ x \in \R^- }
				{
					f''(x) = 6x \leq 0
				}.

				\all{ x \in \R^+ }
				{
					f''(x) = 6x \geq 0
				}.

				f $ is concave over $ \R^- $ and convex over $ \R^+
			}.
			
			\step{ $ Graphic representation $ }
			{
			
				.\includeimage{image1}
			
			}.


			
				\b{I} \all{ x \in (-\infty, - \frac{ \sqrt{ 3 } }{ 2 })}
				{
					\step{$Induction over $ n }
					{
						f $ incresing $ \imp f(x_n) < f(-\frac{ \sqrt{ 3 } }{ 2 }).

						x_{n+1} \in (-\infty, - \frac{ \sqrt{ 3 } }{ 2 })
					}.

					\conclude o(x) $ is enclosed in $ (-\infty, - \frac{ \sqrt{ 3 } }{ 2 }).
			

					\step{ $Induction over $ n }
					{
						x_n^2 > \frac{ 3 }{ 4 } \imp ( x_n^2 - \frac{ 3 }{ 4 }) > 0.

						x_{n+1} - x_n = x_n^3 - \frac{ 3 }{ 4 }x_n = x_n( x_n^2 - \frac{ 3 }{ 4 }) < 0

					}.
					
					\conclude o(x) $ decreasing $.

					\nexists \;  x < - \frac{ \sqrt{ 3 } }{ 2 } \suchthat x $ fixed point $ \imp o(x) \convergesto - \infty				
				}.






				\b{II} \all{ x \in (-\frac{ \sqrt{ 3 } }{ 2 }, 0 ) }
				{
					\step{ $Induction over $ n }
					{
						f $ increasing $ \imp f(-\frac{ \sqrt{ 3 } }{ 2 }) < f(x_n) < f(0).

						x_{n+1} \in (-\frac{ \sqrt{ 3 } }{ 2 }, 0 )
					}.

					\conclude o(x) $ is enclosed in $ (-\frac{ \sqrt{ 3 } }{ 2 }, 0 ).

					\step{ $Induction over $ n }
					{
						x_n^2 < \frac{ 3 }{ 4 } \imp ( x_n^2 - \frac{ 3 }{ 4 }) < 0.

						x_{n+1} - x_n = x_n^3 - \frac{ 3 }{ 4 }x_n = x_n( x_n^2 - \frac{ 3 }{ 4 }) > 0

					}.

					\conclude o(x) $ increasing $.

					o(x) $ convergent $ \logicAnd 0 $ fixed point $ \imp o(x) \convergesto 0
				}.
			

				\b{III}\all{ x \in (0,\frac{ \sqrt{ 3 } }{ 2 }) }
				{
					\step{ $ Induction over $ n }
					{
						-x_n \in (-\frac{ \sqrt{ 3 } }{ 2 }, 0) .

						\b{II} \imp f(-x_n) \in (-\frac{ \sqrt{ 3 } }{ 2 }, 0) \logicAnd f(-x_n) > -x_n.

						f $ odd $ \imp f(x_n) = -f(-x_n) \in (0,\frac{ \sqrt{ 3 } }{ 2 }).

						f $ odd $ \imp f(x_n) = -f(-x_n) < x_n
					}.

					\conclude o(x) $ is enclosed in $ (0, \frac{ \sqrt{ 3 } }{ 2 } ) \logicAnd o(x) $ decreasing $.

					o(x) $ convergent $ \logicAnd 0 $ fixed point $ \imp o(x) \convergesto 0
					
				}.


				\b{IV}\all{ x \in \R }[ x > \frac{ \sqrt{ 3 } }{ 2 } ]
				{
					\step{ $ Induction over $ n }
					{
						-x_n \in ( \frac{ \sqrt{ 3 } }{ 2 }, \infty).

						\b{I} \imp f(-x_n) \in ( \frac{ \sqrt{ 3 } }{ 2 }, \infty) \logicAnd f(-x_n) < -x_n.

						f $ odd $ \imp f(x_n) = -f(-x_n) \in ( \frac{ \sqrt{ 3 } }{ 2 }, \infty).

						f $ odd $ \imp f(x_n) = -f(-x_n) > x_n
					}.

					\conclude o(x) $ is inf bounded by in $ \frac{ \sqrt{ 3 } }{ 2 } \logicAnd o(x) $ increasing $.

					o(x) $ convergent $.

					\nexists \;  x > \frac{ \sqrt{ 3 } }{ 2 } \suchthat x $ fixed point $ \imp o(x) \convergesto +\infty

				}	

		}
	}
}




\end{document}